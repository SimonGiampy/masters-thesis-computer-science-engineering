
% ABSTRACT IN ENGLISH
\chapter*{Abstract}

This thesis explores the development of a cost-effective, autonomous mobile manipulation robot designed 
to address complex tasks in industrial and agricultural environments. The system integrates a mobile platform with
a robotic arm, leveraging the ROS2 robotics framework, Nav2 for autonomous navigation, and MoveIt2 for motion planning. 
A primary focus is the design and implementation of a pneumatic soft gripper, supported by 3D-printed components, 
to enable delicate object manipulation.
The project addresses the challenges of creating a fully autonomous system capable of operating in dynamic,
unstructured environments. This involves the fusion of various sensor data, the development 
of robust algorithms for perception and object manipulation, and the seamless integration of diverse hardware
and software components. Rigorous testing in both simulated and real-world scenarios, mimicking realistic 
agricultural and industrial settings, ensures the system's practicality and effectiveness.

Key contributions include a modular software architecture that facilitates future developments to the project,
software libraries for perception, and a successful demonstration of complex 
tasks involving both robots. The project also investigates the use of a soft gripper,
highlighting its advantages in handling delicate items with its deformable nature.
The thesis successfully answers the questions posed at its outset, demonstrating the feasibility of constructing 
a capable mobile manipulator by combining existing robotic platforms. The solution demonstrates a cost-effective 
approach to automation, leveraging affordable components and open-source software frameworks. 
It showcases the system's ability to perform intricate tasks, such as picking and placing objects,
navigating autonomously, and adapting to dynamic environments.

The results of this research hold significant promise for a variety of applications in agriculture and industry.
The developed mobile manipulator can enhance productivity, efficiency, and safety by automating repetitive and potentially 
hazardous tasks. Furthermore, the modular design provides a foundation for future 
research and development in this field, fostering the creation of even more adaptable and intelligent robotic systems.

\textbf{Keywords}: Mobile Manipulation, Autonomous Navigation, Soft Robotics, ROS2, MoveIt2

% ABSTRACT IN ITALIAN
\chapter*{Abstract in lingua italiana}

Questa tesi esplora lo sviluppo di un robot mobile per la manipolazione, autonomo ed economicamente vantaggioso,
progettato per affrontare compiti complessi in ambienti industriali e agricoli. Il sistema integra una piattaforma 
mobile con un braccio robotico, sfruttando il framework ROS2, Nav2 per la navigazione autonoma e 
MoveIt2 per la pianificazione del movimento. Un focus primario è la progettazione e l'implementazione di un gripper 
pneumatica morbido, supportato da componenti stampati in 3D, per consentire la manipolazione di oggetti delicati.
Il progetto affronta le sfide della creazione di un sistema completamente autonomo in grado di operare in ambienti dinamici
e non strutturati. Ciò implica la fusione di dati provenienti da diversi sensori, lo sviluppo di algoritmi
per la percezione e la manipolazione di oggetti, e l'integrazione di diverse componenti hardware e software. 
Test rigorosi in scenari sia simulati che reali, che riproducono contesti agricoli e industriali realistici, 
assicurano la praticità e l'efficacia del sistema.

I contributi chiave includono un'architettura software modulare che facilita futuri sviluppi al progetto, librerie 
software per la percezione e una dimostrazione di successo di compiti complessi che coinvolgono entrambi i robot. 
Il progetto indaga anche l'uso di un gripper morbido, evidenziandone i vantaggi nella manipolazione di oggetti delicati
grazie alla sua natura deformabile.
La tesi risponde con successo alle domande poste all'inizio, dimostrando la fattibilità di costruire un robot mobile
capace combinando piattaforme robotiche esistenti. La soluzione dimostra un approccio efficace
all'automazione, sfruttando componenti accessibili e framework open-source. Mette in mostra la capacità del
sistema di eseguire compiti articolati, come raccogliere e posizionare oggetti, navigare autonomamente e 
adattarsi ad ambienti dinamici.

I risultati di questa ricerca sono promettenti per una varietà di applicazioni in agricoltura e nell'industria. 
Il robot mobile sviluppato può migliorare la produttività, l'efficienza e la sicurezza automatizzando compiti
ripetitivi e potenzialmente pericolosi. Inoltre, il design modulare fornisce una base per future ricerche e
sviluppi in questo campo, favorendo la creazione di sistemi robotici ancora più adattabili e intelligenti.

{
\begin{adjustwidth}{0cm}{-0.1cm} % Slightly reduce margins
\textbf{Parole chiave}: Manipolazione Mobile, Navigazione Autonoma, Robot Soft, ROS2, MoveIt2
\end{adjustwidth}
}