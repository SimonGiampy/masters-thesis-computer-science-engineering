
\chapter{Introduction}

\section{Context}

Robots were originally designed to assist or replace humans by performing repetitive
and/or dangerous tasks which humans usually prefer not to do, or are unable to do
because of physical limitations imposed by extreme environments. Those include the
limited accessibility of narrow, long pipes underground, anatomical locations in specific
minimally invasive surgery procedure, objects at the bottom of the sea, for example.
With the continuous developments in mechanics, sensing technology, intelligent control
and other modern technologies, robots have improved autonomy capabilities and are
more dexterous.

Nowadays, commercial and industrial robots are in widespread use with
lower long-term cost and greater accuracy and reliability, in the 2 fields like manufacturing,
assembly, packing, transport, surgery, earth and space exploration, etc.
Articulated robots, are among the most common robots used today. They look like a human arm and
that is why they are also called robotic arm or manipulator arm. In some contexts,
a robotic arm may also refer to a part of a more complex robot. A robotic arm can be
described as a chain of links that are moved by joints which are actuated by motors \cite{liu2021deep}.

The majority of robotics applications focus either on navigation aspects of mobile platforms
(e.g. industrial transportation systems, guide robots), or the manipulation of goods
with robotic arms (e.g., bin-picking applications). Nonetheless, few applications consider mobile manipulation
itself combining both robotic tasks. Despite there are several commercial mobile manipulators
in the market, there is a lack of real applications due to the complexity and uncertainty introduced
by combining both, manipulation and navigation \cite{liu2021deep}.

Due to the particular morphology of robotic arms, their scope is limited, and not all the positions of the
base near the table enable a successful picking. Traditionally, such mobile manipulation operations have
been solved using analytical planning and control methods. These methods require explicit programming of the
skills which can be very costly and error-prone particularly in problems where decision making is complex.
The performance of these models depends on how well the reality fits the assumptions made by the model.
Due to the impossibility of predicting all the cases that may occur in dynamic
and unstructured environments, these methods are generally impractical \cite{liu2021deep}.

Traditionally, well known planning and control methods have been widely used for scheduling
mobile manipulation behaviours, for example using the ROS navigation stack for navigation and SLAM
and MoveIt! for arm and object manipulation.

\section{Contribution}

todo: write contribution of the thesis work

\section{Structure}

todo: write structure: list chapters and their content briefly


