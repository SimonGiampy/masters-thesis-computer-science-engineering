\chapter*{Introduction}

Robots were originally designed to perform repetitive tasks
and/or dangerous tasks for humans in extreme environments.
With continuous developments in mechanics, sensing technology, intelligent control,
and other modern technologies, robots have improved autonomy capabilities and are more dexterous.

Articulated robots, also called robotic arms or manipulator arms, are among the most common robots used today. 
In some contexts, a robotic arm may also refer to a part of a more complex robot. A robotic arm can be
described as a chain of links that are moved by joints that are actuated by motors.

The majority of robotics applications focus either on navigation aspects of mobile platforms
(e.g. industrial transportation systems, guide robots), or the manipulation of goods
with robotic arms (e.g., bin-picking applications). Nonetheless, few applications consider mobile manipulation
combining both robotic tasks. There is a lack of real applications of mobile manipulation systems
due to the complexity and uncertainty introduced by combining both manipulation and navigation.

Traditionally, such mobile manipulation operations have been solved using analytical planning and control methods. 
These methods require explicit programming of the skills which can be very costly and error-prone, particularly
in problems where decision-making is complex and the environment dynamic and partially known.
The performance of these models depends on how well the real world fits the assumptions made by the model.
Well-known planning and control methods have been widely used for mobile manipulation behaviors, 
for example using the Nav2 navigation framework, SLAM algorithms for localization and navigation,
and MoveIt for arm and object manipulation.

The main challenge in mobile manipulation is to combine the navigation and manipulation tasks in a single system.
The navigation task requires the robot to move from one place to another, while the manipulation task requires the robot
to interact with objects in the environment. The robot must be able to plan and execute both tasks in a coordinated manner.
This requires the robot to have a good understanding of the environment, including the location of objects and obstacles,
and the ability to plan and execute complex manipulation tasks. 

This Thesis project aims to develop a mobile manipulation system that performs manipulation tasks in a
dynamic environment. The system is based on two robots: a mobile robot and a robotic arm. The mobile robot is equipped
with a LiDAR sensor for navigation and the robotic arm is equipped with a camera for object detection and perception.
The system can perform mobile manipulation tasks in both agricultural and industrial environments.
The entire project revolves around the development of software components that will enable the robots to perform
high-level tasks autonomously, without human intervention, and minimal human supervision.
The focus of the project is the complete autonomy of the system, including navigation, object detection, manipulation,
and task planning.

The project is based on already available
robotic platforms and hardware, and the focus will be on the development of software components.
This project includes the development of algorithms, software packages, and libraries that will be
used to control the mobile manipulation system and perform high-level tasks.

The final objective of the project is the development and realization of two demonstrations that will showcase the capabilities
of the mobile manipulation system. The first demonstration is a system that can interact with a control panel in 
industrial environments, specifically pressing buttons on a box knowing only the relative positioning of the buttons
and the ArUco markers on the box. The second demonstration is a system that can interact with a plant in an agricultural
environment, specifically picking up fruits from an artificial tree. 

This Thesis is structured as follows:

\begin{itemize}
    \item Chapter 1: \textbf{State of the Art and Literature Review}. 
    This chapter provides an overview of the state-of-the-art and a review of the relevant literature on mobile manipulation.
    \item Chapter 2: \textbf{Robotic Platform for Mobile Manipulation}.
    This chapter describes the robotic platform used in the project, including all the hardware components and sensors.
    \item Chapter 3: \textbf{Software Architecture and Simulation environments}.
    This chapter describes the software architecture of the system, all the algorithms, software libraries, and the simulation 
    environments used for testing and development of the system.
    \item Chapter 4: \textbf{Experimental Setup and Demonstrations}.
    This chapter describes the experimental setups used to test the system and demonstrate the capabilities of the entire system.
    \item Chapter 5: \textbf{Results and Future Work}.
    This chapter presents the results of the experiments and discusses the limitations of the system and future work.
\end{itemize}