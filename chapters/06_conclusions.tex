\chapter{Conclusions and Future Work}

This thesis has explored the challenges and potential solutions for developing a mobile manipulator robot
capable of operating in dynamic, unstructured environments. The primary goal is to demonstrate the feasibility 
of autonomous navigation and object manipulation using various end-effector configurations, not to surpass human capabilities, 
but to establish a foundation for future research and more complex robotic systems. This work has addressed the
need for adaptable and intelligent robots that can perform a wide range of tasks in real-world scenarios,
such as in agriculture, and industries, using affordable and accessible hardware and software tools.

Key contributions of this work include the implementation of a modular software framework using ROS2, enabling seamless 
integration of diverse functionalities such as perception, planning, and control. A novel approach to visual 
end effector servo-ing, though not complete, showcases the potential for enhancing object manipulation capabilities. 
The successful integration of a pneumatic soft gripper, while highlighting the complexities of modeling its deformable nature,
demonstrated the value of such grippers in handling delicate objects. This feature is particularly relevant for
applications in agriculture, where the ability to grasp and manipulate fragile produce is essential.

The experience gained in developing this system has emphasized the critical importance of robust perception and localization 
in dynamic environments, particularly concerning sensor data fusion and data processing. The project also underscores 
the challenges of integrating different hardware components and the need for meticulous calibration and configuration.
It also highlights the importance of developing a comprehensive simulation environment to test and validate the system
before deployment in the real world, which can help to identify potential issues in a safer environment 
and improve the system's performance and reliability. The most valuable lesson learned from this project is the importance
of a systematic and iterative approach to robot development, starting from simple tasks in a simulated scenario
and gradually increasing complexity to ensure the system's adaptability and scalability to real-world applications.

The mobile manipulator system presented in this thesis serves as a proof of concept 
for automating tasks commonly found in agricultural and industrial settings. The modular software design and comprehensive
documentation provide a valuable starting point for future researchers to extend and refine the system's capabilities.
The current implementation shows the limitations of the system in terms of reliability, particularly in the grasping
and manipulation of objects, which can be improved through more sophisticated grasping strategies and control algorithms.

The legacy of this project lies in its potential to inspire further research and development in the field of mobile
manipulation. Future work could focus on enhancing more advanced perception algorithms
and exploring deep learning techniques to improve object recognition and grasping strategies.
A key area for improvement is the integration of ROS2 behavior trees for more complex task planning and execution,
enabling more articulated and adaptive behaviors in the robots.

Ultimately, this research aims to contribute to the broader goal of creating adaptable and intelligent robotic systems 
capable of performing complex tasks in real-world environments, using low-cost robotic platforms and open-source software.
The project showcases the potential of mobile manipulators to revolutionize industries such as agriculture, industrial automation,
and logistics, by providing flexible and cost-effective solutions to a wide range of applications. By addressing the challenges
and limitations of the current system, future research can build upon this work to develop more sophisticated and capable
robotic systems with the use of imitation learning and deep reinforcement learning techniques to learn complex manipulation tasks.
These advancements will pave the way for a new generation of robots that can operate autonomously in dynamic and unstructured
environments, transforming the way we work and live in the future.
