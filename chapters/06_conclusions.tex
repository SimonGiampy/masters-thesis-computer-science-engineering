\chapter{Conclusions and Future Work}

This thesis has presented the development of a mobile manipulator robot capable of performing complex tasks in 
dynamic and unstructured environments. The primary objective of this work is to develop a system that can
autonomously navigate and manipulate objects with different end effector configurations in unstructured environments.
The goal is not to develop a system that works better and faster than humans but to show that an autonomous robotic
system can perform complex tasks that are usually handled by humans. The system is designed to be the working
ground for future research and more complex projects in the field of mobile manipulation.

The agricultural and industrial environments are the main targets for the developed system. Environments such as these
ones are usually dynamic and unstructured, and the tasks that need to be performed are usually repetitive.
This makes them ideal for automation, even though their complexity poses a challenge to the development of autonomous
systems, because the reliability and robustness of the system are crucial for their success.
The demonstrations presented and illustrated in the project, are 

The entire mobile manipulation system developed throughout the project is a proof of concept for the automation of
tasks that are very easy for humans to perform, and common in agricultural and industrial environments,
where the system finds its main applications and use cases. Automating tasks in the agricultural field can lead to significant
increases in productivity and efficiency, especially in environments where the tasks are repetitive and where
there is room for improvement in terms of resource management. Automating tasks in industrial plants instead
can be useful for increasing the safety of the workers, by removing them from dangerous environments, and for
increasing the efficiency of the production lines by diagnosing problems with the sensors mounted on the robot.

The project has been developed with the idea of being a working ground for future research and more complex projects
in the field of mobile manipulation. The code has been written with modularity in mind so that it can be easily
extended and modified to fit the needs of future projects. Several ROS2 packages have been developed as 
a library of tools and functionalities that can be easily integrated into other projects. 

The legacy of this project is thus the codebase and the extensive documentation that has been written throughout
the development of the project. The codebase is a collection of ROS2 packages that can be used as a starting point
for future projects and as a reference for the development of new functionalities. The documentation is a collection
of notes and explanations that can be used to replicate the demonstrations and to understand the codebase,
and how to extend it to create new functionalities. The ROS2 controller for the robotic arm is also a legacy of
this project, as it is the first controller for the robotic arm that has been developed in ROS2 and that can be
operated via ethernet. The hardware setup used for the soft gripper and the pneumatic system is also a legacy of
this project, as it is a working setup that can be integrated into other projects with ease and minimal effort.

% Future work section

Future developments of the project are mainly focused on the improvement of the system in terms of robustness and
reliability, especially in the perception and manipulation software modules. The perception module can be improved by
using more advanced algorithms for object detection and tracking, for example by training the neural network
on a larger dataset and by using more advanced architectures using a GPU for inference. The manipulation module
can be improved by taking into consideration the mechanics and dynamics of the end effector so that the grasping
and manipulation of objects can be more reliable and robust to objects with irregular shapes and different sizes.

The greatest possibilities for improvement are in the autonomous control of the robot arm and the end effector.
The current implementation of the controller is very basic and based on traditional methods with open-source
libraries for trajectory planning and control. The controller can be improved by using more advanced methods
such as deep reinforcement learning or imitation learning techniques, which can be used to learn the control
policy of the entire mobile manipulation system for more complex tasks. The autonomous control with the
methods used throughout the project is limited to tasks that are can be explicitly programmed, and it is not
suitable for tasks that require a high level of dexterity and precision. Instead, future work can focus on
developing a more advanced controller for more complex tasks that are difficult or impossible to be
programmed explicitly. 